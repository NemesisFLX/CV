\documentclass{resume}

\usepackage{hyperref}
\usepackage{fontawesome}
\usepackage{wasysym} 
\usepackage{tikz}

\newcommand{\roundpic}[4][]{
  \tikz\node [circle, minimum width = #2,
    path picture = {
      \node [#1] at (path picture bounding box.center) {
        \includegraphics[width=#3]{#4}};
    }] {};}

\begin{document}

\fontfamily{ppl}\selectfont

\noindent
\begin{tabularx}{\linewidth}{@{}m{0.8\textwidth} m{0.2\textwidth}@{}}
{
    \Large{Felix Mönckemeyer} \newline
    \small{
        \clink{
            \href{mailto:felix.moenckemeyer@gmail.com}{felix.moenckemeyer@gmail.com} \textbf{·} 
            {\fontdimen2\font=0.75ex +49 176 2354 9970}
            \textbf{·} 
            {\fontdimen2\font=0.75ex Köln, Germany}
        } 
        \begin{flushleft}
            \footnotesize Hochmotivierter \textbf{Software Entwickler} (IoT \& Fullstack) mit großem Interesse an komplexen Problemenstellungen. Mehr als 2 Jahre Erfahrung in der Entwicklung von industrieller Software von der Maschine bis in die Cloud. \textbf{Projektleitung}, Anforderungsanalyse, Planung, Testing und Dokumentation.
        \end{flushleft}
    }
} & 
{
    \hfill
    \roundpic[]{4cm}{4cm}{images/portrait.jpeg}
}
\end{tabularx}
\vspace{-5mm}
\begin{center}
\small "Ich möchte meine Fähigkeiten in der Softwareentwicklung erweitern und enger mit dem Kunden an echten Lösungen arbeiten, die einen direkten Mehrwert bieten. "
%\vspace{5mm}

%Softwareentwickler für Int Anwendungen mit acht Jahren Erfahrung in der Entwicklung einer breiten Palette von Tools für iOS und Android für eine Reihe von Kunden. 
%Ich habe ausgewiesene Expertise in der Entwicklung von datenmanagement Applikationen. 
%Ich verstehe den Lebenszyklus von Backendsoftware sehr gut und bin sehr fähig in allen Aspekten der Entwicklung, von der Projektplanung über die Anforderungserfassung bis hin zum Schreiben und Testen von Code, Erstellen von Dokumentation. 
%Ich bin derzeit auf der Suche nach einer langfristigen Position, die es mir ermöglicht, meine Fähigkeiten im Projektmanagement weiter zu verbessern.

\vspace{5mm} 
\begin{tabularx}{\linewidth}{@{}*{2}{X}@{}}
% left side %
{
    \csection{ERFAHRUNG}{\small
        \begin{itemize}
            % item 1 %
            \item \frcontent{Senseering GmbH}{Full Stack Software Engineer - Köln}{
            Entwickler einer dezentralen data sharing platform im Sinne von \href{https://www.bmwi.de/Redaktion/DE/Dossier/gaia-x.html}{GAIA-X}. Software Berater - IoT Infrastruktur Lösungen.}{seit April 2020}
             \item \frcontent{WZL der RWTH Aachen}{Werkstudent - Aachen}{Entwickler einer Cloud und DLT basierten Datatrading Plattform.}{April 2018 - April 2020}
            
        \end{itemize}
    }
    \csection{AUSBILDUNG}{\small
        \begin{itemize}
            % item 1 %
            \item \frcontent{B.Sc. Informatik \footnotesize }{Rheinisch-Westfälische Technische Hochschule Aachen}{}{Oktober 2014 - Oktober 2018}
            \item \frcontent{M.Sc. Informatik \footnotesize }{Rheinisch-Westfälische Technische Hochschule Aachen}{}{Seit Oktober 2018}
        \end{itemize}
    }
} 
% end left side %
& 
% right side %
{
    \csection{SKILLS}{\small
        \begin{itemize}
             \item \textbf{Programmiersprachen \& Datenbanken} \newline
            {\footnotesize Nodejs, Vue, Python, MySQL, Mongodb, Influxdb, Redis}
            \item \textbf{Cloud \& Technologien} \newline
            {\footnotesize S3/Azure Blob Storage, EC2, CodeBuild, CodePipeline, CloudFront, Grafana, docker, Node Red, Github Actions, DynamoDB, AWS Cognito, Auth0, IOTA, Nginx, Route 53, ELB, API Gateway, AWS Lambda}
              \item \textbf{Sprachen} \newline
            {\footnotesize Deutsch, Englisch}
        \end{itemize}
    }
    \csection{PROJEKTE}{\small
        \begin{itemize}
            \item \frcontent{spaicer \clink{\href{https://www.spaicer.de/}{[spaicer.de]}}}{Skalierbare adaptive Produktionssysteme durch KI-Basierte Resilienzoptimierung - Projektverantwortung }{}{Projektverantwortung, IoT, Machine Learning, KI}
            \item \frcontent{obsidian \clink{\href{https://github.com/Senseering/obsidian}{[Senseering/obsidian]}}}{A Nodejs based immutability layer for (industrial) data}{Release pending}{Nodejs, IOTA}
             \item \frcontent{MyDataEconomy \clink{\href{https://www.mydataeconomy.com}{[mydataeconomy.com]}}}{Decentralized IoT data sharing platform for sovereign data exchange.}{}{GAIA-X, Nodejs, Docker, IOTA, InfluxDB}
        \end{itemize}
    }
}
\end{tabularx}
\end{center}
\vspace{20mm}
\begin{center}

\begin{tabularx}{\linewidth}{@{}*{3}{X}@{}}
\centering{\href{https://www.linkedin.com/in/felix-m\%C3\%B6nckemeyer-6887a61a0}{ \Large  \faLinkedinSquare } }
&
\centering{ \href{https://github.com/NemesisFLX}{\Large \faGithub } }
&{\hspace{25mm}\href{mailto:felix.moenckemeyer@gmail.com}{\Large \faEnvelope }}
\\\centering\small Felix Mönckemeyer &
\centering\small NemesisFLX  & 
\centering\small felix.moenckemeyer@gmail.com
\end{tabularx}

\end{center}
\end{document}